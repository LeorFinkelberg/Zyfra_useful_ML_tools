\documentclass[%
	11pt,
	a4paper,
	utf8,
	%twocolumn
		]{article}	

\usepackage{style_packages/podvoyskiy_article_extended}


\begin{document}
\title{Заметки по машинному обучению и анализу данных}

\author{\itshape Подвойский А.О.}

\date{}
\maketitle

\thispagestyle{fancy}

Здесь будут собираться заметки по вопросам, касающимся машинного обучения,
интеллектуального анализа данных, программирования (Python, Scala etc.), прикладной математики, а также по прочим сопряженным вопросам так или иначе, затрагивающим работу с данными.


\shorttableofcontents{Краткое содержание}{1}

\tableofcontents

\section{Вопросы}



\listoffigures\addcontentsline{toc}{section}{Список иллюстраций}

% Источники в "Газовой промышленности" нумеруются по мере упоминания 
\begin{thebibliography}{99}\addcontentsline{toc}{section}{Список литературы}
	\bibitem{beazley:python-2010}{\emph{Бизли Д.} Python. Подробный справочник. -- Пер. с англ. -- СПб.: Символ-Плюс, 2010. -- 864~с. }
\end{thebibliography}

\end{document}
